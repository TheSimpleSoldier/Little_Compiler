\documentclass[12pt]{article}
\usepackage{parskip}
\usepackage[T1]{fontenc}

\title{A little Compiler}
\author{Joshua Leger, Fred Kneeland}
\date{Febuary 9, 2016}

\begin{document}
    \maketitle
    \begin{abstract}
	
		The goal of this project was to learn the fundamentals of compilation through the implementation of working compiler for the little language.  This project was done through Montana State University as the senior capstone project of the professional option for Computer Science.  	
	
    \end{abstract}
    \clearpage
    \tableofcontents
    \clearpage
    
    \section{Introduction}
		Compilers are a critical and foundation piece of computer science.  They are the tools that allowed for the abstraction of higher level languages, allowing far greater speed of development as well as rapid implementation of tedious optimizations.  As compilers are such critical tools in software development, learning both the theory and practical implementations behind them is an important aspect of any computer science education.
		    
    \section{Background}
    		// TODO: complete this section
    	
    \section{Methods and Discussion}
    
    			This project was implemented in python and run from the command line.  A file was given to the program as an argument and the compiler would then parse that file.
    
    	\subsection{Scanner}
    			For the Scanner we utilized an open source python lexer by Dabeaz LLC.  We then used some regex statements to determine if an expression was valid and used that to validate the syntax and read in the expressions to be parsed. The following are the regular expressions used.

                First is the string literal:

                \begin{verbatim}
                    "[^"]*"
                \end{verbatim}

                This matches a quotation mark followed by 0 or more non quotation mark characters, ending in another quotation mark.

                Next are operators:

                \begin{verbatim}
                    <=|>=|:=|\+|-|\*|/|=|!=|<|>|\(|\)|;|,
                \end{verbatim}

                This was simply a large or statement, looking for each possible operator. Escape characters had to be used for +, *, (, and ). The <= and >= also had to come before < and > so it did not interpret <= as the < operator followed by the = operator.

                Float literals are defined as:

                \begin{verbatim}
                    [0-9]*\.[0-9]+
                \end{verbatim}

                Floats have 0 or more digits followed a period followed by 1 or more digits.

                Similarly, integer literals are defined as: 

                \begin{verbatim}
                    [0-9]+
                \end{verbatim}

                Integers are simply a list of 1 or more digits.

                Keywords and identifiers are defined with the same regular expression:

                \begin{verbatim}
                    [a-zA-Z][a-zA-Z0-9]*
                \end{verbatim}

                Identifiers are simply a letter followed by 0 or more letters and numbers. To identify keywords, a check on the string is run for equality to any of the keywords defined in the list named reserved defined at the top of the file.

                Finally, comments are defined as:

                \begin{verbatim}
                    --.*
                \end{verbatim}

                This matches a -- followed by the rest of the line. Instead of assigning a token to this, it is simply skipped over.

                Everything else is simply skipped over. This includes all whitespace.

                The order to search for tokens is string literal, float literal, int literal, operator, identifier, and keyword. Float has to be in front of int, but otherwise the order of checking does not matter.




    	\subsection{Parser}
    		For the parser we used the ply library.  With this library, yacc (Yet another Compiler Compiler) was used to do the manual labor of parsing through the tokens, once the grammar was provided.   In this stage of the project the formal definition of the Little Language was converted into a python grammar that the yacc parser could read in.  Using the Lexer created above, the parser was feed a string of tokens which it then evaluated as it built a parse tree.  
    		
    		We used the following structure to define a program: 
    		\begin{verbatim}
                    'statement : PROGRAM IDENTIFIER BEGIN pgm_body END'
           \end{verbatim}
           
           A program in the little language is composed of the keyword "PROGRAM" a name and then a code block in between the key words 'BEGIN' and 'END'
           
           The program body was defined as follows:
           \begin{verbatim}
                   '''pgm_body : decl func_decl'''
           \end{verbatim}
           
           Basically the program is divided into two sections.  The first is a list of global variables and the second section is all of the functions.
           
           the function list was defined as follows in such a way as to allow any number of functions including zero.
           
           \begin{verbatim}
           		'''func_decl : func_declaration func_decl
               	             | empty'''
            \end{verbatim}
            
            Similar to most c languages function declarations were composed of a return type, an indentifier, a list of parameters and a function body.
            
            \begin{verbatim}
                   '''func_declaration : FUNCTION any_type IDENTIFIER LEFTPAREN param_decl_list RIGHTPAREN BEGIN func_body END '''
           \end{verbatim}
           
           The paramater list is defined as a list of variable types and names separated by commas.
           
            \begin{verbatim}
                   '''param_decl_list : param_decl param_decl_tail
                                              | empty'''
           \end{verbatim}
           
			A param decl is where parameters name and type is specified.
			
			\begin{verbatim}
                   '''param_decl : var_type IDENTIFIER'''
           \end{verbatim}
			            
           The param decl tail was used to ensure that the variables were seperated by Commas and that the list could be extended.
           
           \begin{verbatim}
                   '''param_decl_tail : COMMA param_decl param_decl_tail
					                           | empty'''
           \end{verbatim}
           
           The func body is very similar to the program body as it is composed of two pieces, variable declarations and statements.
           
           \begin{verbatim}
                   '''func_body : decl_func_var stmt_list'''
           \end{verbatim}
           
           Inside of stmt list is where most of the heavy lifting of the program takes place.  Statements include all of the operations, conditional statments and loops.
           
            \begin{verbatim}
                  '''stmt_list : stmt stmt_list
				                    | empty '''
				   '''stmt : base_stmt
             				   | if_stmt
		               	   | while_stmt'''
           \end{verbatim}
           
           Base Statements are used for all of the standard one line operations such as assigning values, or expressions, to variables, returning values, reading, and writing.
           \begin{verbatim}
                   '''base_stmt : assign_stmt
					                    | read_stmt
					                    | write_stmt
					                    | return_stmt'''
           \end{verbatim}
           
           Assignment statements are used for handling operations and various types of expressions.
           
           \begin{verbatim}
           		'''assign_stmt : assign_expr SEMICOLON'''
           		'''assign_expr : IDENTIFIER STRINGEQUALS expr'''
            \end{verbatim}		
           
			Expression statements allow for any combination of variable operators.  
			
			\begin{verbatim}
				"expr : expr_prefix factor"      
				'''expr_prefix : expr_prefix factor addop
                                   | empty'''
			\end{verbatim}	   
			
			Factors represent various kinds of values.  They can be variables, constants or function calls.
			
			\begin{verbatim}
				'''factor : factor_prefix postfix_expr'''
				'''factor_prefix : factor_prefix postfix_expr mulop
				                       | empty'''
				'''postfix_expr : primary
			                          | call_expr'''  
			    "call_expr : IDENTIFIER LEFTPAREN expr_list RIGHTPAREN "                                          
			\end{verbatim}	
			
			Statements can also be conditional statements.
			\begin{verbatim}
				"if_stmt : IF LEFTPAREN cond RIGHTPAREN decl_block_var stmt_list else_part ENDIF"
           \end{verbatim}	
           
           Additionally, statements can become loops.
           \begin{verbatim}
				"while_stmt : WHILE LEFTPAREN cond RIGHTPAREN decl_block_var stmt_list ENDWHILE"
           \end{verbatim}
           
    		
    		The Little language is a very simple language with a few strict rules which significantly reduce complexity.  Each program has a program definition with a body that is composed of two sections.  The first section defines variables and the second is for functions.  The functions themselves have a very limited subset of options with conditional systems and only a while loop.  Therefore an unambiguous grammar for this language proved to be a relatively trivial task.  
    		

    	\subsection{Symbol Table}
    		Now that the compiler parses through the tokens the next step is to create a symbol table.  For this aspect of the project we created a couple of data structures.  First we had a variable object which would hold information about a specific variable including its name and value.  Additionally, we created a symbol table object.  This object had a parent, a list of children and a list of variables.  Furthermore, we also stored the current scope level.  Thus we could navigate up the scope list to find a declaration of a variable or we could add another scope as a child if we had a function or Block statement.  
    		
    		The main difficulty for constructing the symbol table was from ply.  When parsing the leaves were executed in reverse order.  This meant that the parser would find a variable declaration before it knew what symbol table the declaration was suppose to be a part of.   To counteract this, temporary symbol tables were created and when the declaration of a function or Block statement was hit these were then appended to the correct symbol table.  The same was true of individual variables and in order to meet the requirement that variables appear in the proper order in the symbol table, variables needed logic about whether to be inserted at the beginning or end of the variable list in order to maintain their proper orientation relative to one another.

    	\subsection{Semantic Routines}
    		TODO

    	\subsection{Full fledged Compiler}
    		TODO

    \section{Conclusion and future work}	
			TODO	    
    
 \end{document}
